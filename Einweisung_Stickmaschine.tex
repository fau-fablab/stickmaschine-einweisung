%%%%%%%%%%%%%%%%%%%%%%%%%%%%%%%%%%%%%%%%%%%%%%%%
% COPYRIGHT: (C) 2012-2015 FAU FabLab and others
% Bearbeitungen ab 2015-02-20 fallen unter CC-BY-SA 3.0
% Sobald alle Mitautoren zugestimmt haben, steht die komplette Datei unter CC-BY-SA 3.0. Bis dahin ist der Lizenzstatus aller alten Bestandteile ungeklärt.
%%%%%%%%%%%%%%%%%%%%%%%%%%%%%%%%%%%%%%%%%%%%%%%%


\newcommand{\basedir}{fablab-document}
\documentclass{\basedir/fablab-document}

\usepackage{amssymb} % Symbole für Knöpfe
\usepackage{subfigure,caption}

\usepackage{marvosym} % für Briefumschlag-Symbol
\usepackage{eurosym}
\usepackage{tabularx} % Tabellen mit bestimmtem Breitenverhältnis der Spalten
\usepackage{multirow} % Tabellen Zellen die sich über mehrere Zeilen ausdehnen
\usepackage{wrapfig} % Textumlauf um Bilder
\renewcommand{\texteuro}{\euro}

\linespread{1.2}

\date{2022}
\author{Harald Kraft u.a.}
\fancyfoot[L]{kontakt@fablab.fau.de}
\title{Einweisung Stickmaschine Brother VR}

%auskommentierte Zeilen aus Lasercutter-Einweisung, als evtl. Vorlage im Dokument behalten
%\tikzstyle{laserknopf} = [anchor=base, draw=black, fill=gray!10, rectangle, rounded corners, inner sep=2pt, outer sep = 3pt]
%
%% styles für das Flowchart
%\tikzstyle{decision} = [diamond, draw, fill=blue!20,
%    text width=4.5em, text badly centered, node distance=3cm, inner sep=0pt]
%\tikzstyle{block} = [rectangle, draw, fill=blue!20,
%    text width=5em, text centered, rounded corners, minimum height=4em]
%\tikzstyle{line} = [draw, very thick, color=black!50, -latex']
%\tikzstyle{cloud} = [draw, ellipse,fill=red!20, node distance=3cm,
%    minimum height=2em]
%
%% Knöpfe für Laser und Fernbedienung
%\newcommand{\knopf}[2]{
%    \begin{tikzpicture}[baseline={(box.base)}]
%    \node [#1] (box) {
%        \fontsize{9pt}{9pt}\selectfont \textbf{#2}\strut
%    };
%    \end{tikzpicture}
%}


\renewcommand{\todo}[1]{\textbf{\color{red}{TODO: #1}}}
\newcommand{\pfeil}{\ensuremath{\rightarrow}}

\usepackage{hyperref}
	\hypersetup{
	colorlinks   = true,    % Colours links instead of ugly boxes
	urlcolor     = blue,    % Colour for external hyperlinks
	linkcolor    = blue,    % Colour of internal links
	citecolor    = red      % Colour of citations
}
\newcommand*{\fullref}[1]{\hyperref[{#1}]{\autoref*{#1} \nameref*{#1}}}

\begin{document}

\maketitle
\todo{Das ist aktuell noch größtenteils die Einweisung für die alte Stickmaschine. Diese Einweisung ist daher noch nicht für den Produktiveinsatz freigegeben!}

\section{Regeln und Hinweise}
\subsection{Was du dir merken solltest}
\begin{itemize}
	\item nichts mit Gewalt verschieben, einsetzen, bewegen, verstellen! Die Mechanik ist leichtgängig bei korrekter Handhabung und Behandlung.
	\item für jede Stickarbeit die richtige und einwandfreie (gerade, nicht abgebrochen) Nadel verwenden
	\item nicht bei beschädigter Stichplatte nähen (sonst droht der Nadelbruch) und an die Betreuer wenden.
	\item ohne Rücksprache mit einem Betreuer darf niemals die Unterfadenspannung geändert werden.
	\item immer die Maschine beobachten) und ggf. Stop drücken bzw. die Maschine ausschalten, falls irgendetwas untypisch ist (Geräusche, Geruch, Nadelbewegung...)
	\item beim Sticken immer ein Vlies verwenden
\end{itemize}

\subsection{Welche Fragen du beantworten / Was du vorführen können solltest}
\begin{itemize}
	\item Oberfaden einfädeln
	\item Unterfadenspule aufspulen
	\item Unterfadenspule einsetzen
	\item Stoff in Stickrahmen einspannen
	\item kleine Stickrahmen in die Maschine einsetzen
	\item große Stickrahmen in die Maschine einsetzen
	\item 
\end{itemize}
\vspace{5em}
\hrule

\section{Sticksoftware Hatch Embroidery 2}

Zur Vorbereitung 

\subsection{Sticklaptop}

	Im Fablab steht ein Laptop mit der o.g. Sticksoftware bereit. bitte befolgende folgende Schritte, um die Software richtig zu nutzen:

	\begin{enumerate}
		\item Sticklaptop \textcolor{red}{vor(!)} dem Einschalten mit einem Netzwerkabel an eine freie Netzwerkdose im Fablab anstecken
		\item Netzkabel des Laptops einstecken
		\item Laptop einschalten
		\item Wenn du gefragt wirst, ob du CAD Nutzen möchtest bitte auf \textcolor{red}{JA} klicken
		\item Sticksoftware 'Hatch 2' starten
	\end{enumerate}

\subsection{Stoffart einstellen}

\subsection{Stickmuster anpassen}

\subsection{Sticharten anpassen}

\subsection{Text einfügen}

\subsection{Bilder einfügen und in Stickmuster umwandeln}

\pagebreak

\section{Sticken mit der Stickmaschine Brother VR}

\subsection{Vor dem Sticken}
\begin{itemize}
	\item Stickmaschine auf dem Hubtisch betreiben oder auf dem einen stabilen Tisch stellen, sodass sie \textbf{nicht wackelt}
	\item falls notwendig, Höhenausgleich der Füße
	\item Netzkabel anstecken
	\item Greifer ölen (siehe \fullref{sec:oelen})
\end{itemize}

\subsection{Nach dem Sticken}
\begin{itemize}
	\item Stickmaschine abschalten, bevor das Netzkabel getrennt wird
	\item nach dem Sticken, immer den Stickrahmen entnehmen
	\item falls die Maschine von dem Hubwagen genommen wurde, bitte wieder darauf stellen
	\item das Zubehör bitte wieder dahin zurückräumen, wo es hingehört
	\item die Haube zum Schutz der empfindlichen Teile wieder auf die Maschine setzen
	\item den Wagen mit der Stickmaschine wieder an ihren vorgesehenen Platz zurückfahren
\end{itemize}

\subsection{Teile der Maschine und Zubehör}

Eine Abbildung mit der Benennung aller Teile der Maschine findest du in der Gebrauchsanleitung auf S. 14f. Lese dir diese durch und verinnerliche die Namen der Maschinenteile, um diese Einweisung zu verstehen.


\subsection{Nadeln}
\textcolor{red}{Wichtig:} Unbedingt einwandfreie Nadeln verwenden: gute Qualität, gerade, nicht abgebrochen, schlägt nicht an, usw.!

Die Maschine ist für die Verwendung von Flachkolben-Sticknadeln ausgelegt. Vom Hersteller werden Nadeln vom Typ Organ HAX 130 EBBR empfohlen. Alternativ können auch Schmetz 130/705 H-E verwendet werden. 
 
Die Verwendung falscher Nadeln kann zu Nadelbrüchen, Fadenrissen oder Schaden am Nadeleinfädlermechanismus führen.

\textcolor{red}{Verwende daher stets nur vom Fablab zur Verfügung gestellete Nadeln oder kläre die Verwendung von anderen Nadeln immer vorher mit den Betreuer:innen ab.}

\vspace{2em}

\textbf{Nadel wechseln:}
\newline \textbf{Anleitung S.72 oder zeigen lassen}
\begin{itemize}
	\item Stickmaschine ausschalten
	\item Nadel in höchste Stellung mit Handrad bringen
	\item Schraube rechts an Nadel mit Zubehörwerkzeug öffnen
	\item Nadel entfernen (nicht einfach herausfallenlassen, sondern herausnehmen)
	\item Nadel mit abgeflachter Seite nach hinten einlegen
	\item Nadel bis zum Anschlag nach oben schieben
	\item Schraube festdrehen -  fest, aber ohne Gewalt
	
\end{itemize}

\subsection{Greifer ölen}
\label{sec:oelen}
\begin{itemize}
	\item Greiferklappe öffnen
	\item 
\end{itemize}

\vspace{2em}

\subsection{Oberfaden einfädeln}

Bei Metallgarnen oder ähnlich starken Garnen kann es sinn machen vor dem Sticken eines der beiliegenden Spulennetze über die Garnrollen zu ziehen. Wenn das Netz zu lang ist, falte es einmal, um es auf die richtige Längezu bringen.
Bei Verwendung von Garnrollennetzen muss möglicherweise die Fadenspannung eingestellt werden.

\textbf{Ausführliche Anleitung S.45 und zeigen lassen}
\begin{enumerate}
	\item Wenn noch nicht vorhanden, Filzscheibe auf Garnrollenhalten legen
	\item Garnrolle aufstecken
	\item Garnrollenkappe aufstecken
	\item Faden durch Loch in Metalleiste über Garnrollenhalter fädeln (Nr. 1)
	\item Faden durch Loch auf Maschinenoberseite führen (Nr. 2)
	\item von links unter erste Führungsplatte 
	\item im Uhrzeigersinn einmal um den Fadenspanner wickeln (Nr. 3)
	\item von links unter zweite Führungsplatte ziehen
	\item im rechten Schlitz nach unten, unten herum und im linken Schlitz nach oben führen (Nr. 4)
	\item von recht durch Fadenhebel führen (Nr. 5)
	\item im linken Schlitz nach unten führen und durch Loch (Nr. 6) ziehen
	\item von Hinten in Drahtfadenführung an Nadelstange einhängen
	\item Einfädeltaste (ganz rechts) drücken, damit der automatische Einfädler nach vorne fährt
	\item ca. 15cm Faden herausziehen
	\item von rechts unter die Gabel des automatischen Einfädlers führen
	\item Faden von links nach oben über Fadenabschneider führen, um ihn auf die richtige länge zu kürzen und dort hängen lassen
	\item Einfädeltaste (ganz rechts) eneut drücken drücken, damit der faden eingefädelt wird
	\item kontrollieren, ob Faden richtig eingefädelt ist
\end{enumerate}

\vspace{2em}

\subsection{Unterfaden einsetzen}
\textcolor{red}{Als Unterfaden ist üblicherweise weißes und schwarzes Garn auf großen Konen vorhanden. Bitte nicht das deutlich teurere farbige Stickgarn verwenden, da der Unterfaden sowieso nicht sichtbar sein wird.}

\vspace{1em}
\textbf{Ausführliche Anleitung S.32 oder zeigen lassen}
\begin{enumerate}
	\item Greiferabdeckung herunterklappen \textcolor{red}{Bitte Vorsicht}: Klappe kann im heruntergeklappten Zustand leicht abbrechen)
	\item Fadenaufnehmer ohne Gwalt ein Stück nach vorne klappen. \textcolor{red}{Achtung: wird immer ein Stück nach oben stahen - ann nie ganz heruntergeklappt werden}
	\item Hebel an Spulenkapsel nach vorne ziehen, so festhalten und Kapsel herausziehen 
	\item Falls noch eine Spule in der Spulenkapsel vorhanden ist: Spule herausnehmen
	\item Greifer ölen (siehe Kapitel \fullref{sec:oelen})
	\item Falls keine bereits aufgespulte Unterfadenspule vorhanden ist, muss eine neue aufgespult werden (siege Kapitel ''Unterfaden aufspulen'')
	\item Spule so einlegen, so dass sich der Faden gegen den Unterzeigersinn abrollt
	\item Faden in den Schlitz und unter Metallfeder führen
	\item Faden ca. 5cm aus der Spule herausziehen
	\item Hebel an Spulenkapsel nach hinten ziehen, damit Spule nicht herausfallen kann und wieder vollständig in Greifer einsetzen
	\item Greiferabdeckung wieder nach oben klappen
\end{enumerate}

\subsection{Unterfaden aufspulen}
\textbf{Ausführliche Anleitung S.34 oder zeigen lassen}
\begin{enumerate}
	\item passende Spulen verwenden
	\item Spule auf den Spuler drücken
	\item Faden nach Anleitung einfädeln und einige Male per Hand um die Spule wickeln oder durch das Loch in der Spule fädeln
	\item Spulenhebel gegen die Spule drücken \pfeil Spule dreht sich bis sie voll ist oder manuell Hebel wegdrücken
	\item Faden abschneiden, Spule abnehmen
\end{enumerate}

\subsection{Vorbereitungen Maschine}
\begin{enumerate}
	\item Nähmaschine ausschalten oder ausgeschalten lassen
	\item Stickfuß anbringen (siehe oben)
	\item Sticknadel (Embroidery) einsetzen
	\item Zubehörfach entfernen
	\item Abdeckung zum Anschluss des Stickaggregats öffnen, Transporteur versenken (siehe oben)
	\item Stickaggregat anbringen und einrasten
	\item Maschine mit USB Kabel mit Laptop verbinden und Laptop hochfahren
	\item Unterfaden farblich passendes normales Nähgarn
	\item Oberfaden Stickgarn je nach Farbwunsch (Einfädeln siehe oben) \pfeil \textcolor{red}{Wichtig: darauf achten, dass der Faden richtig eingerastet ist und unter Spannung steht}
\end{enumerate}

\subsection{Vorbereitungen Stoff, Einspannen}
\begin{itemize}
	\item Entscheidung Größe des Designs: kleiner Stickrahmen oder großer Stickrahmen
	\item Rahmenstellschraube öffnen und Rahmen auseinander nehmen
	\item normalerweise bei \textbf{nicht dehnbaren} Stoffen:
	\begin{itemize}
		\item Stickvlies etwas größer als den gewählten Rahmen ausschneiden
		\item Vlies unten und Stoff darüber zwischen die beiden Rahmenteile legen (Beim großen Rahmen müssen die beiden Pfeile aufeinander treffen!)
		\item Stoff fest einspannen, dazu am Stoff ziehen (Vorsicht bei Vlies, dieses kann reißen)
		\item beim Klopfen auf den Stoff soll ein Geräusch ähnlich einer Trommel sein
		\item \textcolor{red}{Achtung, dass der Stoff richtig positioniert ist!} 
	\end{itemize}
\end{itemize} 


\subsection{Vliesauswahl}
\begin{itemize}
	\item \textbf{normales Stickvlies, reißbar:} \textbf{immer} bei nicht dehnbaren Stoffen unter den Stoff in den Rahmen einspannen, kann nach dem Sticken abgerissen werden
	\item \textbf{normales Stickvlies, nicht reißbar, fest:} bei gewünschter extra Stabilität bei nicht dehnbaren Stoffen, muss nach dem Sticken zurückgeschnitten werden
	\item \textbf{Bügelvlies, nicht reißbar:} kann bei dehnbaren Stoffen verwendet werden oder wenn Vlies nicht den ganzen Rahmen ausfüllen soll, Vlies vor dem Einspannen des Stoffes in den Stickrahmen mit einem Bügeleisen auf den Stoff aufbringen. Bügelanleitung (\textcolor{red}{Temperatur!}) des Vlieses und des zu bestickenden Stoffes beachten  
	\item \textbf{Klebevlies, reißbar:}kann bei dehnbaren Stoffen verwendet werden oder bei Stoffstücken kleiner als der Rahmen, nur Vlies einspannen, mit Stecknadel vorsichtig die Folie anritzen (nicht das eigentliche Vlies) und abziehen, Stoff positionieren und glatt aufkleben, kann nach dem Sticken abgerissen werden
	\item \textbf{Stickvlies bzw. -folie wasserlöslich:} \textbf{nur} Nach dem Sticken kann Vlies/Folie durch Befeuchten aufgelöst werden, zB Sticken von Spitze
	Auch bei voluminösen Stoffen (zB Frottee, Fleece): hier werden 3 Schichten eingespannt: Vlies, Stoff, Stickfolie dann wird gestickt und die oberste Schicht Folie nach dem Sticken durch Befeuchten entfernt
so wird das Einsinken der Fäden in den Stoff verhindert!
\end{itemize}

\subsection{Stickmuster auf Maschine laden}
\begin{itemize}
	\item
\end{itemize}

\subsection{Sticken}
\textcolor{red}{Immer Maschine beobachten und \textbf{Stop} drücken, falls irgendetwas untypisch ist!}

\vspace{1em}

\begin{itemize}
	\item Nähfuß anheben
	\item Nadel in höchste Position
	\item Stickrahmen einlegen: Hebel am Stickfuß nach oben drücken und Rahmen in die Führung schieben, sodass der Stoff oben liegt und sich nicht verhängen kann, muss einrasten
	\item Nähfuß senken
	\item Faden leicht festhalten, damit er bei den ersten Stichen nicht durchrutscht
	\item Speed ungefähr in der Mitte, Start drücken, wenige Stiche sticken lassen, Stop drücken, Faden abschneiden, Start drücken
	\item Maschine stickt alles einer Farbe selbstständig und regelt die Geschwindigkeit und schneidet am Ende eines Blocks den Faden selbstständig ab
\end{itemize}


\subsection{Nach dem Sticken}
\begin{itemize}
	\item Nähfuß anheben
	\item Hebel links vom Rahmen am Stickaggregat zum Lösen des Rahmens drücken
	\item Rahmen nach vorne schieben und Stickfuß am Hebel weiter anheben, damit er über den Rahmen kommt
	\item Stoff aus Rahmen lösen und Vlies/Folie entfernen
	\item Stickaggregat mit Hebel an der Vorderseite lösen und entfernen
	\item Transporteur wieder anheben
	\item Abdeckung Stecker schließen
	\item normales Zubehörfach anbringen
	\item Nadel herausnehmen
	\item benutzte Nadel mit Lackstift markieren
\end{itemize}

% schoenere Darstellung
\newpage
\ccLicense{stickmaschine-einweisung}{Einweisung Stickmaschine}

\end{document}